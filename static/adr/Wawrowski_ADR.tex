\documentclass[]{book}
\usepackage{lmodern}
\usepackage{amssymb,amsmath}
\usepackage{ifxetex,ifluatex}
\usepackage{fixltx2e} % provides \textsubscript
\ifnum 0\ifxetex 1\fi\ifluatex 1\fi=0 % if pdftex
  \usepackage[T1]{fontenc}
  \usepackage[utf8]{inputenc}
\else % if luatex or xelatex
  \ifxetex
    \usepackage{mathspec}
  \else
    \usepackage{fontspec}
  \fi
  \defaultfontfeatures{Ligatures=TeX,Scale=MatchLowercase}
\fi
% use upquote if available, for straight quotes in verbatim environments
\IfFileExists{upquote.sty}{\usepackage{upquote}}{}
% use microtype if available
\IfFileExists{microtype.sty}{%
\usepackage{microtype}
\UseMicrotypeSet[protrusion]{basicmath} % disable protrusion for tt fonts
}{}
\usepackage[margin=1in]{geometry}
\usepackage{hyperref}
\hypersetup{unicode=true,
            pdftitle={Metody przetwarzania i analizy danych w R},
            pdfauthor={Łukasz Wawrowski},
            pdfborder={0 0 0},
            breaklinks=true}
\urlstyle{same}  % don't use monospace font for urls
\usepackage{natbib}
\bibliographystyle{plainnat}
\usepackage{color}
\usepackage{fancyvrb}
\newcommand{\VerbBar}{|}
\newcommand{\VERB}{\Verb[commandchars=\\\{\}]}
\DefineVerbatimEnvironment{Highlighting}{Verbatim}{commandchars=\\\{\}}
% Add ',fontsize=\small' for more characters per line
\usepackage{framed}
\definecolor{shadecolor}{RGB}{248,248,248}
\newenvironment{Shaded}{\begin{snugshade}}{\end{snugshade}}
\newcommand{\KeywordTok}[1]{\textcolor[rgb]{0.13,0.29,0.53}{\textbf{#1}}}
\newcommand{\DataTypeTok}[1]{\textcolor[rgb]{0.13,0.29,0.53}{#1}}
\newcommand{\DecValTok}[1]{\textcolor[rgb]{0.00,0.00,0.81}{#1}}
\newcommand{\BaseNTok}[1]{\textcolor[rgb]{0.00,0.00,0.81}{#1}}
\newcommand{\FloatTok}[1]{\textcolor[rgb]{0.00,0.00,0.81}{#1}}
\newcommand{\ConstantTok}[1]{\textcolor[rgb]{0.00,0.00,0.00}{#1}}
\newcommand{\CharTok}[1]{\textcolor[rgb]{0.31,0.60,0.02}{#1}}
\newcommand{\SpecialCharTok}[1]{\textcolor[rgb]{0.00,0.00,0.00}{#1}}
\newcommand{\StringTok}[1]{\textcolor[rgb]{0.31,0.60,0.02}{#1}}
\newcommand{\VerbatimStringTok}[1]{\textcolor[rgb]{0.31,0.60,0.02}{#1}}
\newcommand{\SpecialStringTok}[1]{\textcolor[rgb]{0.31,0.60,0.02}{#1}}
\newcommand{\ImportTok}[1]{#1}
\newcommand{\CommentTok}[1]{\textcolor[rgb]{0.56,0.35,0.01}{\textit{#1}}}
\newcommand{\DocumentationTok}[1]{\textcolor[rgb]{0.56,0.35,0.01}{\textbf{\textit{#1}}}}
\newcommand{\AnnotationTok}[1]{\textcolor[rgb]{0.56,0.35,0.01}{\textbf{\textit{#1}}}}
\newcommand{\CommentVarTok}[1]{\textcolor[rgb]{0.56,0.35,0.01}{\textbf{\textit{#1}}}}
\newcommand{\OtherTok}[1]{\textcolor[rgb]{0.56,0.35,0.01}{#1}}
\newcommand{\FunctionTok}[1]{\textcolor[rgb]{0.00,0.00,0.00}{#1}}
\newcommand{\VariableTok}[1]{\textcolor[rgb]{0.00,0.00,0.00}{#1}}
\newcommand{\ControlFlowTok}[1]{\textcolor[rgb]{0.13,0.29,0.53}{\textbf{#1}}}
\newcommand{\OperatorTok}[1]{\textcolor[rgb]{0.81,0.36,0.00}{\textbf{#1}}}
\newcommand{\BuiltInTok}[1]{#1}
\newcommand{\ExtensionTok}[1]{#1}
\newcommand{\PreprocessorTok}[1]{\textcolor[rgb]{0.56,0.35,0.01}{\textit{#1}}}
\newcommand{\AttributeTok}[1]{\textcolor[rgb]{0.77,0.63,0.00}{#1}}
\newcommand{\RegionMarkerTok}[1]{#1}
\newcommand{\InformationTok}[1]{\textcolor[rgb]{0.56,0.35,0.01}{\textbf{\textit{#1}}}}
\newcommand{\WarningTok}[1]{\textcolor[rgb]{0.56,0.35,0.01}{\textbf{\textit{#1}}}}
\newcommand{\AlertTok}[1]{\textcolor[rgb]{0.94,0.16,0.16}{#1}}
\newcommand{\ErrorTok}[1]{\textcolor[rgb]{0.64,0.00,0.00}{\textbf{#1}}}
\newcommand{\NormalTok}[1]{#1}
\usepackage{longtable,booktabs}
\usepackage{graphicx,grffile}
\makeatletter
\def\maxwidth{\ifdim\Gin@nat@width>\linewidth\linewidth\else\Gin@nat@width\fi}
\def\maxheight{\ifdim\Gin@nat@height>\textheight\textheight\else\Gin@nat@height\fi}
\makeatother
% Scale images if necessary, so that they will not overflow the page
% margins by default, and it is still possible to overwrite the defaults
% using explicit options in \includegraphics[width, height, ...]{}
\setkeys{Gin}{width=\maxwidth,height=\maxheight,keepaspectratio}
\IfFileExists{parskip.sty}{%
\usepackage{parskip}
}{% else
\setlength{\parindent}{0pt}
\setlength{\parskip}{6pt plus 2pt minus 1pt}
}
\setlength{\emergencystretch}{3em}  % prevent overfull lines
\providecommand{\tightlist}{%
  \setlength{\itemsep}{0pt}\setlength{\parskip}{0pt}}
\setcounter{secnumdepth}{5}
% Redefines (sub)paragraphs to behave more like sections
\ifx\paragraph\undefined\else
\let\oldparagraph\paragraph
\renewcommand{\paragraph}[1]{\oldparagraph{#1}\mbox{}}
\fi
\ifx\subparagraph\undefined\else
\let\oldsubparagraph\subparagraph
\renewcommand{\subparagraph}[1]{\oldsubparagraph{#1}\mbox{}}
\fi

%%% Use protect on footnotes to avoid problems with footnotes in titles
\let\rmarkdownfootnote\footnote%
\def\footnote{\protect\rmarkdownfootnote}

%%% Change title format to be more compact
\usepackage{titling}

% Create subtitle command for use in maketitle
\newcommand{\subtitle}[1]{
  \posttitle{
    \begin{center}\large#1\end{center}
    }
}

\setlength{\droptitle}{-2em}

  \title{Metody przetwarzania i analizy danych w R}
    \pretitle{\vspace{\droptitle}\centering\huge}
  \posttitle{\par}
    \author{Łukasz Wawrowski}
    \preauthor{\centering\large\emph}
  \postauthor{\par}
    \date{}
    \predate{}\postdate{}
  
\usepackage{booktabs}
\usepackage{amsthm}
\makeatletter
\def\thm@space@setup{%
  \thm@preskip=8pt plus 2pt minus 4pt
  \thm@postskip=\thm@preskip
}
\makeatother

\begin{document}
\maketitle

{
\setcounter{tocdepth}{1}
\tableofcontents
}
\chapter*{Wprowadzenie}\label{wprowadzenie}
\addcontentsline{toc}{chapter}{Wprowadzenie}

Literatura podstawowa:

\begin{itemize}
\tightlist
\item
  Przemysław Biecek -
  \href{http://pbiecek.github.io/Przewodnik/}{\emph{Przewodnik po
  pakiecie R}}
\item
  Marek Gągolewski -
  \href{http://www.gagolewski.com/publications/programowanier/}{\emph{Programowanie
  w języku R. Analiza danych, obliczenia, symulacje.}}
\item
  Garret Grolemund, Hadley Wickham -
  \href{http://r4ds.had.co.nz/}{\emph{R for Data Science}}
  (\href{https://helion.pl/ksiazki/jezyk-r-kompletny-zestaw-narzedzi-dla-analitykow-danych-hadley-wickham-garrett-grolemund,jezrko.htm}{polska
  wersja})
\end{itemize}

Literatura dodatkowa:

\begin{itemize}
\tightlist
\item
  \href{https://github.com/mi2-warsaw/SER/blob/master/histoRia/README.md}{inne
  pozycje po polsku}
\item
  \href{https://bookdown.org/}{inne pozycje po angielsku}
\end{itemize}

Internet:

\begin{itemize}
\tightlist
\item
  \href{https://www.r-bloggers.com/}{R-bloggers}
\item
  \href{https://rweekly.org/}{rweekly}
\end{itemize}

\chapter{Wprowadzenie}\label{wprowadzenie-1}

\section{Narzędzie}\label{narzedzie}

\begin{itemize}
\tightlist
\item
  darmowe
\item
  wszechstronne
\item
  wsparcie społeczności
\item
  wersja desktopowa i serwerowa
\end{itemize}

czyli \textbf{R} - środowisko do obliczeń statystycznych i wizualizacji
wyników

\begin{itemize}
\tightlist
\item
  strona projektu: \href{https://www.r-project.org/}{r-project.org}
\item
  świetne IDE: \href{https://www.rstudio.com/}{RStudio}
\item
  wersja przeglądarkowa: \href{https://rstudio.cloud/}{rstudio.cloud}
\end{itemize}

\href{https://www.business-science.io/business/2018/10/08/python-and-r.html}{R
+ Python}

\section{Cele analiz}\label{cele-analiz}

Podstawowe:

\begin{itemize}
\tightlist
\item
  wnioskowanie statystyczne - porównywanie grup
\item
  regresja - poszukiwanie związków
\item
  klasyfikacja - przyporządkowanie do grup
\item
  grupowanie - poszukiwanie grup
\item
  prognozowanie - patrzenie w przyszłość
\end{itemize}

Inne:

\begin{itemize}
\tightlist
\item
  analiza języka naturalnego
\item
  rozpoznawanie obrazów
\item
  analiza koszykowa
\item
  \ldots{}
\end{itemize}

\subsection{Eksporacja danych}\label{eksporacja-danych}

Pakiet \texttt{tidyverse}

\begin{Shaded}
\begin{Highlighting}[]
\KeywordTok{library}\NormalTok{(tidyverse)}
\end{Highlighting}
\end{Shaded}

\begin{itemize}
\tightlist
\item
  analiza częstości dla zmiennych jakościowych
\item
  analiza struktury dla zmiennych ilościowych
\end{itemize}

Case study: \href{data/wybory2018.xlsx}{Wybory 2018}

\chapter{Testowanie hipotez}\label{testowanie-hipotez}

\section{Hipoteza statystyczna}\label{hipoteza-statystyczna}

Przypuszczenie dotyczące własności analizowanej cechy, np. średnia w
populacji jest równa 10, rozkład cechy jest normalny.

Formułuje się zawsze dwie hipotezy: hipotezę zerową (\(H_0\)) i hipotezę
alternatywną (\(H_1\)). Hipoteza zerowa jest hipotezą mówiącą o
równości:

\(H_0: \bar{x}=10\)

Z kolei hipoteza alternatywna zakłada coś przeciwnego:

\(H_1: \bar{x}\neq 10\)

Zamiast znaku nierówności (\(\neq\)) może się także pojawić znak
mniejszości (\(<\)) lub większości (\(>\)).

\section{Poziom istotności i wartość
p}\label{poziom-istotnosci-i-wartosc-p}

Hipotezy statystyczne weryfikuje się przy określonym poziomie istotności
\(\alpha\), który wskazuje maksymalny poziom akceptowalnego błędu
(najczęściej \(\alpha=0,05\)).

Większość programów statystycznych podaje w wynikach testu wartość
p.~Jest to prawdopodobieństwo uzyskania obserwowanych wyników przy
założeniu prawdziwości hipotezy zerowej.

Generalnie jeśli \(p < \alpha\) - odrzucamy hipotezę zerową.

\href{http://idane.pl/blog/asa}{Krytyka wartości p}

\section{Testy parametryczne i
nieparametryczne}\label{testy-parametryczne-i-nieparametryczne}

Testy statystyczne dzielą się na dwie grupy:

\begin{itemize}
\tightlist
\item
  parametryczne, które wymagają spełnienia założeń, ale są
  dokładniejsze,
\item
  nieparametryczne, które nie wymagają tylu założeń, ale są mniej
  dokładne.
\end{itemize}

\chapter{Regresja}\label{regresja}

\section{Regresja prosta}\label{regresja-prosta}

Na podstawie \href{data/Salary_Data.csv}{danych} dotyczących informacji
o doświadczeniu i wynagrodzeniu pracowników zbuduj model określający
`widełki' dla potencjalnych pracowników o doświadczeniu równym 8, 10 i
11 lat.

\href{res/regresja_prosta.Rmd}{regresja\_prosta.Rmd}

\href{res/adr.zip}{cały projekt}

\subsection{Zadanie}\label{zadanie}

Dla danych dotyczących \href{data/sklep77.csv}{sklepu nr 77} opracuj
model zależności sprzedaży od liczby klientów. Ile wynosi teoretyczna
sprzedaż w dniach, w których liczba klientów będzie wynosiła 560, 740,
811 oraz 999 osób?

\section{Regresja wieloraka}\label{regresja-wieloraka}

Na podstawie danych dotyczących \href{data/pracownicy.csv}{zatrudnienia}
opracuj model, w którym zmienną zależną jest bieżące wynagrodzenie. Jaka
cecha ma największy wpływ na tę wartość?

Opis zbioru:

\begin{itemize}
\tightlist
\item
  id - kod pracownika
\item
  plec - płeć pracownika (0 - mężczyzna, 1 - kobieta)
\item
  data\_urodz - data urodzenia
\item
  edukacja - wykształcenie (w latach nauki)
\item
  kat\_pracownika - grupa pracownicza (1 - ochroniarz, 2 - urzędnik, 3 -
  menedżer)
\item
  bwynagrodzenie - bieżące wynagrodzenie
\item
  pwynagrodzenie - początkowe wynagrodzenie
\item
  staz - staż pracy (w miesiącach)
\item
  doswiadczenie - poprzednie zatrudnienie (w miesiącach)
\item
  zwiazki - przynależność do związków zawodowych (0 - nie, 1 - tak)
\item
  wiek - wiek (w latach)
\end{itemize}

\href{res/regresja_wieloraka.Rmd}{regresja\_wieloraka.Rmd}

\href{res/adr.zip}{cały projekt}

\subsection{Zadanie}\label{zadanie-1}

Na podstawie zbioru dotyczącego \href{data/50_Startups.csv}{50
startupów} określ jakie czynniki w największym stopniu wpływają na
przychód startupów.

\chapter{Grupowanie}\label{grupowanie}

Metody grupowania są wykorzystywane np. do segmentacji klientów, w
przypadku, gdy nie jest znany końcowy podział.

\section{Metoda k-średnich}\label{metoda-k-srednich}

Algorytm:

\begin{enumerate}
\def\labelenumi{\arabic{enumi}.}
\tightlist
\item
  Wskaź liczbę grup \(k\).
\item
  Wybierz dowolne \(k\) punktów jako centra grup.
\item
  Przypisz każdą z obserwacji do najbliższego centroidu.
\item
  Oblicz nowe centrum grupy.
\item
  Przypisz każdą z obserwacji do nowych centroidów. Jeśli któraś
  obserwacja zmieniła grupę - przejdź do kroku nr 4, a w przeciwnym
  przypadku zakończ algorytm.
\end{enumerate}

Zalety:

\begin{itemize}
\tightlist
\item
  dobrze działa zarówno na małych, jak i dużych zbiorach
\item
  efektywny
\end{itemize}

Wady:

\begin{itemize}
\tightlist
\item
  trzeba wskazać liczbę grup
\item
  losowy wybór punktów początkowych
\end{itemize}

\section{Metoda hierarchiczna}\label{metoda-hierarchiczna}

Algorytm:

\begin{enumerate}
\def\labelenumi{\arabic{enumi}.}
\tightlist
\item
  Każda obserwacji stanowi jedną z \(N\) pojedyńczych grup.
\item
  Na podstawie macierzy odległości połącz dwie najbliżej leżące
  obserwacje w jedną grupę (\(N-1\) grup).
\item
  Połącz dwa najbliżej siebie leżące grupy w jedną (\(N-2\) grup).
\item
  Powtórz krok nr 3, aż do uzyskania jednej grupy.
\end{enumerate}

Zalety:

\begin{itemize}
\tightlist
\item
  prosty sposób ustalenia liczby grup
\item
  praktyczny sposób wizualizacji
\end{itemize}

Wady:

\begin{itemize}
\tightlist
\item
  nieodpowiedni dla dużych zbiorów
\end{itemize}

\subsection{Zadanie}\label{zadanie-2}

Na podstawie zbioru zawierającego informacje o
\href{data/klienci.csv}{klientach sklepu} dokonaj grupowania klientów.

Opis zbioru:

\begin{itemize}
\tightlist
\item
  klientID - identyfikator klienta
\item
  plec - płeć
\item
  wiek - wiek
\item
  roczny\_dochod - roczny dochód wyrażony w tys. dolarów
\item
  wskaznik\_wydatkow - klasyfikacja sklepu od 1 do 100
\end{itemize}

\href{res/grupowanie.Rmd}{grupowanie.Rmd}

\href{res/adr.zip}{cały projekt}

\subsection{Zadanie 2}\label{zadanie-2-1}

Dokonaj grupowania danych dotyczących \href{data/auta.csv}{32
samochodów} według następujących zmiennych: pojemność, przebieg, lata
oraz cena.

\subsection{Zadanie 3}\label{zadanie-3}

Rozpoznawanie czynności na podstawie danych z przyspieszeniomierza w
telefonie:
\href{http://archive.ics.uci.edu/ml/datasets/User+Identification+From+Walking+Activity\#}{User
Identification From Walking Activity Data Set}

\chapter{Klasyfikacja}\label{klasyfikacja}

\href{http://www.r2d3.us/visual-intro-to-machine-learning-part-1/}{A
visual introduction to machine learning} - niestety powstała tylko jedna
część.

\section{Drzewa klasyfikacyjne}\label{drzewa-klasyfikacyjne}

Zalety:

\begin{itemize}
\tightlist
\item
  łatwa interpretacja
\item
  nie trzeba normalizować cech
\item
  rozwiązuje problemy liniowe i nieliniowe
\end{itemize}

Wady:

\begin{itemize}
\tightlist
\item
  mała efektywność przy małych zbiorach danych
\item
  łatwo można przeuczyć
\end{itemize}

\section{KNN}\label{knn}

Algorytm:

\begin{enumerate}
\def\labelenumi{\arabic{enumi}.}
\tightlist
\item
  Określ liczbę sąsiadów - \(K\)
\item
  Wyznacz \(K\) sąsiadów dla nowego punktu na podstawie wybranej
  odległości
\item
  Oblicz liczbę sąsiadów, w każdej z grup
\item
  Przypisz nową obserwację do grupy, w której ma więcej najbliższych
  sąsiadów
\end{enumerate}

Zalety:

\begin{itemize}
\tightlist
\item
  łatwa interpretacja
\item
  szybki i efektywny
\end{itemize}

Wady:

\begin{itemize}
\tightlist
\item
  trzeba określić liczbę sąsiadów
\end{itemize}

\subsection{Zadanie}\label{zadanie-4}

Zbuduj model klasyfikacyjny dla zbioru
\href{data/Social_Network_Ads.csv}{danych} dotyczących cech internautów
oraz informacji czy zamówili reklamowany produkt czy nie.

Przeprowadź imputację braków danych dla zbioru
\href{data/pracownicy.RData}{pracowników}.

\chapter{Materiały z zajęć}\label{materiay-z-zajec}

\section{28.10.2018}\label{section}

\href{res/skrypt20181028.R}{Wprowadzenie do R}

\href{res/analiza20181028.R}{Analiza sejmików}

\section{18.11.2018}\label{section-1}

\href{https://departmentofstatisticspue.github.io/statystyka-opisowa/analiza-struktury.html}{Analiza
struktury}

\href{data/rossmann.xlsx}{Rossmann}

\href{res/zajecia20181118.R}{Analiza struktury w R}

\section{16.12.2018}\label{section-2}

\href{prezentacje/03.html}{Prezentacja}

\href{data/pracownicy.csv}{Pracownicy}


\end{document}
