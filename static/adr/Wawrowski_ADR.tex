\documentclass[]{book}
\usepackage{lmodern}
\usepackage{amssymb,amsmath}
\usepackage{ifxetex,ifluatex}
\usepackage{fixltx2e} % provides \textsubscript
\ifnum 0\ifxetex 1\fi\ifluatex 1\fi=0 % if pdftex
  \usepackage[T1]{fontenc}
  \usepackage[utf8]{inputenc}
\else % if luatex or xelatex
  \ifxetex
    \usepackage{mathspec}
  \else
    \usepackage{fontspec}
  \fi
  \defaultfontfeatures{Ligatures=TeX,Scale=MatchLowercase}
\fi
% use upquote if available, for straight quotes in verbatim environments
\IfFileExists{upquote.sty}{\usepackage{upquote}}{}
% use microtype if available
\IfFileExists{microtype.sty}{%
\usepackage{microtype}
\UseMicrotypeSet[protrusion]{basicmath} % disable protrusion for tt fonts
}{}
\usepackage[margin=1in]{geometry}
\usepackage{hyperref}
\hypersetup{unicode=true,
            pdftitle={Metody przetwarzania i analizy danych w R},
            pdfauthor={Łukasz Wawrowski},
            pdfborder={0 0 0},
            breaklinks=true}
\urlstyle{same}  % don't use monospace font for urls
\usepackage{natbib}
\bibliographystyle{plainnat}
\usepackage{longtable,booktabs}
\usepackage{graphicx,grffile}
\makeatletter
\def\maxwidth{\ifdim\Gin@nat@width>\linewidth\linewidth\else\Gin@nat@width\fi}
\def\maxheight{\ifdim\Gin@nat@height>\textheight\textheight\else\Gin@nat@height\fi}
\makeatother
% Scale images if necessary, so that they will not overflow the page
% margins by default, and it is still possible to overwrite the defaults
% using explicit options in \includegraphics[width, height, ...]{}
\setkeys{Gin}{width=\maxwidth,height=\maxheight,keepaspectratio}
\IfFileExists{parskip.sty}{%
\usepackage{parskip}
}{% else
\setlength{\parindent}{0pt}
\setlength{\parskip}{6pt plus 2pt minus 1pt}
}
\setlength{\emergencystretch}{3em}  % prevent overfull lines
\providecommand{\tightlist}{%
  \setlength{\itemsep}{0pt}\setlength{\parskip}{0pt}}
\setcounter{secnumdepth}{5}
% Redefines (sub)paragraphs to behave more like sections
\ifx\paragraph\undefined\else
\let\oldparagraph\paragraph
\renewcommand{\paragraph}[1]{\oldparagraph{#1}\mbox{}}
\fi
\ifx\subparagraph\undefined\else
\let\oldsubparagraph\subparagraph
\renewcommand{\subparagraph}[1]{\oldsubparagraph{#1}\mbox{}}
\fi

%%% Use protect on footnotes to avoid problems with footnotes in titles
\let\rmarkdownfootnote\footnote%
\def\footnote{\protect\rmarkdownfootnote}

%%% Change title format to be more compact
\usepackage{titling}

% Create subtitle command for use in maketitle
\newcommand{\subtitle}[1]{
  \posttitle{
    \begin{center}\large#1\end{center}
    }
}

\setlength{\droptitle}{-2em}
  \title{Metody przetwarzania i analizy danych w R}
  \pretitle{\vspace{\droptitle}\centering\huge}
  \posttitle{\par}
  \author{Łukasz Wawrowski}
  \preauthor{\centering\large\emph}
  \postauthor{\par}
  \date{}
  \predate{}\postdate{}

\usepackage{booktabs}
\usepackage{amsthm}
\makeatletter
\def\thm@space@setup{%
  \thm@preskip=8pt plus 2pt minus 4pt
  \thm@postskip=\thm@preskip
}
\makeatother

\begin{document}
\maketitle

{
\setcounter{tocdepth}{1}
\tableofcontents
}
\chapter*{Wprowadzenie}\label{wprowadzenie}
\addcontentsline{toc}{chapter}{Wprowadzenie}

Literatura podstawowa:

\begin{itemize}
\tightlist
\item
  Przemysław Biecek -
  \href{http://pbiecek.github.io/Przewodnik/}{\emph{Przewodnik po
  pakiecie R}}
\item
  Marek Gągolewski -
  \href{http://www.gagolewski.com/publications/programowanier/}{\emph{Programowanie
  w języku R. Analiza danych, obliczenia, symulacje.}}
\item
  Garret Grolemund, Hadley Wickham -
  \href{http://r4ds.had.co.nz/}{\emph{R for Data Science}}
  (\href{link}{polska wersja})
\end{itemize}

Literatura dodatkowa:

\begin{itemize}
\tightlist
\item
  \href{https://github.com/mi2-warsaw/SER/blob/master/histoRia/README.md}{inne
  pozycje po polsku}
\item
  \href{https://bookdown.org/}{inne pozycje po angielsku}
\end{itemize}

Internet:

\begin{itemize}
\tightlist
\item
  \href{https://www.r-bloggers.com/}{R-bloggers}
\item
  \href{https://rweekly.org/}{rweekly}
\end{itemize}

\chapter{Wprowadzenie do analizy
danych}\label{wprowadzenie-do-analizy-danych}

Podstawowe cele w analizie danych:

\begin{itemize}
\tightlist
\item
  porównanie grup
\item
  prognozowanie
\item
  klasyfikacja
\end{itemize}

Bez względu na cel analizy jest kilka pojęć, które są wspólne.

\section{Hipoteza statystyczna}\label{hipoteza-statystyczna}

Przypuszczenie dotyczące własności analizowanej cechy, np. średnia w
populacji jest równa 10, rozkład cechy jest normalny.

Formułuje się zawsze dwie hipotezy: hipotezę zerową (\(H_0\)) i hipotezę
alternatywną (\(H_1\)). Hipoteza zerowa jest hipotezą mówiącą o
równości:

\(H_0: \bar{x}=10\)

Z kolei hipoteza alternatywna zakłada coś przeciwnego:

\(H_1: \bar{x}\neq 10\)

Zamiast znaku nierówności (\(\neq\)) może się także pojawić znak
mniejszości (\(<\)) lub większości (\(>\)).

\section{Poziom istotności i wartość
p}\label{poziom-istotnosci-i-wartosc-p}

Hipotezy statystyczne weryfikuje się przy określonym poziomie istotności
\(\alpha\), który wskazuje maksymalny poziom akceptowalnego błędu
(najczęściej \(\alpha=0,05\)).

Większość programów statystycznych podaje w wynikach testu wartość
p.~Jest to prawdopodobieństwo uzyskania obserwowanych wyników przy
założeniu prawdziwości hipotezy zerowej.

Generalnie jeśli \(p < \alpha\) - odrzucamy hipotezę zerową.

\href{http://idane.pl/blog/asa}{Krytyka wartości p}

\section{Testy parametryczne i
nieparametryczne}\label{testy-parametryczne-i-nieparametryczne}

Testy statystyczne dzielą się na dwie grupy:

\begin{itemize}
\tightlist
\item
  parametryczne, które wymagają spełnienia założeń, ale są
  dokładniejsze,
\item
  nieparametryczne, które nie wymagają tylu założeń, ale są mniej
  dokładne.
\end{itemize}

\chapter{Regresja prosta}\label{regresja-prosta}

Na podstawie \href{data/Salary_Data.csv}{danych} dotyczących informacji
o doświadczeniu i wynagrodzeniu pracowników zbuduj model określający
`widełki' dla potencjalnych pracowników o doświadczeniu równym 8, 10 i
11 lat.

\chapter{Regresja wieloraka}\label{regresja-wieloraka}

Na podstawie danych dotyczących \href{data/pracownicy.csv}{zatrudnienia}
opracuj model, w którym zmienną zależną jest bieżące wynagrodzenie. Jaka
cecha ma największy wpływ na tę wartość?


\end{document}
